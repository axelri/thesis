\section{Introduction}

In this report a proof of the game Lumines's NP-completeness is presented. First the rules of the game are formally defined. Using these definitions the NP-complete subset sum problem is reduced to a Lumines gameboard with a defined sequence of blocks. A proof is presented showing that the only way to clear a specific number of blocks in the given Lumines gameboard is if the subset sum instance is a "yes"-instance.

\subsection{Problem statement}
This report considers the generalized version of the game, which is played on an $n \times m$-gameboard. The authors seek to formalize and  simplify Lumines, and explore this new system's computational complexity characteristics. Certain goals of the game are considered by translating them into decision problems in this system.

While the sweep-line is a core characteristic of Lumines, this report considers all marked terrain as instantly cleared. Furthermore the whole sequence of blocks is considered to be known beforehand. This will be referred to as the \textit{offline} version of the game, in contrast to the \textit{online} version, where the blocks are generated probabilistically. These are the two main simplifications, although a more thorough discussion is provided in \nameref{method}. The impact of these simplifications on the validity of the results is considered in \nameref{discussion}.

The main Lumines goals this paper examines are:
\begin{itemize}
        \item Is (standard mode) Lumines NP-complete?
        \item Are all game modes of Lumines equally hard?
\end{itemize}

\subsection{Motivation}

Lumines is a game of interest because to the authors' knowledge it's not well researched from the perspective of computational complexity. Several similar games has been more thoroughly researched in recent years and comparing Lumines to these findings might yield interesting results. New findings on the computational complexity of Lumines may aid in the development of video game artificial intelligence. Our understanding of computational complexity in generic 2-dimensional grid games may also benefit from this research. Combining results from this report with results from previous similar research might make it possible to find a correlation between certain game characteristics and classes of computational complexity.
