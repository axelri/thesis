\subsubsection{Phase 5}
When the open well has collapsed the last phase is entered. In this phase the remaining $\sum M_w - 2$ blocks are placed. The purpose of this phase is to both clear the remaning black cells in column 4, and clear the remaining white cells in column 3 and 4 by making them meet. Let $i = 0$ when block $\sum Q -  \sum M_w + 2$ is about to the placed, let $i = i +1$ when a block has been placed in column 4-5. Then the following invariants hold for the open well $w$:

\begin{enumerate}
\item Column 5 is empty
\item The rows in column 1 are alternating white-black (from the bottom). The rows in column 2 are alternating black-white.
\item In the interval below, the rows in column 3 are white and the rows in column 4 are black:
\begin{equation*}
    \left[1, \left( K - 2 -i \right)\right]
\end{equation*}
\item In the interval below, the rows in column 3 are alternating black-white and the rows in column 4 are white:
\begin{equation*}
    \left[2 \left( K-2-i\right) +1, 2 \left(\sum M_w + K -4 -2i \right)\right]
\end{equation*}
\item In the interval below, the rows in column 3 are alternating black-white and the rows in column 4 are alternating white-black:
\begin{equation*}
    \left[ 2 \left(\sum M_w + K -4 -2i \right) +1, 2 \left( \sum Q + K - 3 -i\right) + 1 \right]
\end{equation*}
\item In the interval below, the rows in column 3 are alternating white-black and the rows in column 4 are empty:
\begin{equation*}
    \left[ 2 \left( \sum Q + K - 2 -i\right), 2 \left( \sum Q + K - 2 -i \right)+1 \right]
\end{equation*}
\item In the interval below, the rows in column 3 and 4 are empty:
\begin{equation*}
    \left[ 2 \left( \sum Q + K - 1 -i\right), 2 \left( \sum Q + K \right) \right]
\end{equation*}

\end{enumerate}

This invariants are pictured in~\autoref{fig:invariantsfive}

\begin{figure}[H]
    \centering
    \resizebox{!}{0.4\paperheight}{
    \begin{tikzpicture}
        \collapselast{0}{0}
        \draw[dashed] (-1, 0) -- (6, 0);
        \draw[dashed] (-1, 6) -- (6, 6);
        \draw[dashed] (-1, 12) -- (6, 12);
        \draw[dashed] (-1, 17) -- (6, 17);
        \draw[dashed] (-1, 19) -- (6, 19);
        \draw[dashed] (-1, 25) -- (6, 25);
        \draw[dashed] (5, 0) -- (5, 25);
        \node at (0.5, -0.5) {\large 1};
        \node at (1.5, -0.5) {\large 2};
        \node at (2.5, -0.5) {\large 3};
        \node at (3.5, -0.5) {\large 4};
        \node at (4.5, -0.5) {\large 5};
        \draw [decorate,decoration={brace,amplitude=10pt},xshift=-12pt,yshift=0pt]
        (0, 0) -- (0, 6) node [left,align=right,black,midway,xshift=-1cm]
        {\huge $2 \left( K-2-i \right)$ rows};

        \draw [decorate,decoration={brace,amplitude=10pt},xshift=-12pt,yshift=0pt]
        (0, 6) -- (0, 12) node [left,align=right,black,midway,xshift=-1cm]
        {\huge $2 \left( \sum M_w - 2 -i \right)$ rows};

        \draw [decorate,decoration={brace,amplitude=10pt},xshift=-12pt,yshift=0pt]
        (0, 12) -- (0, 17) node [left,align=right,black,midway,xshift=-1cm]
        {\huge $2 \left( \sum Q - \sum M_w + i + 1 \right) +1$ rows};

        \draw [decorate,decoration={brace,amplitude=10pt},xshift=-12pt,yshift=0pt]
        (0, 19) -- (0, 25) node [left,align=right,black,midway,xshift=-1cm]
        {\huge $3+2i$ rows};

        \draw (3.5, 0.5) -- (7, 0.5) node [right, black, xshift=1cm] 
        {\huge row 1};

        \draw (3.5, 5.5) -- (7, 5.5) node [right, black, xshift=1cm, yshift=-0.5cm] 
        {\huge row $2 \left( K-2-i \right)$};

        \draw (3.5, 6.5) -- (7, 6.5) node [right, black, xshift=1cm, yshift=0.5cm] 
        {\huge row $2 \left( K-2-i \right) + 1$};

        \draw (3.5, 11.5) -- (7, 11.5) node [right, black, xshift=1cm, yshift=-0.5cm] 
        {\huge row $2 \left( \sum M_w + K - 4 -2i \right)$};

        \draw (3.5, 12.5) -- (7, 12.5) node [right, black, xshift=1cm, yshift=0.5cm] 
        {\huge row $2 \left( \sum M_w + K - 4 -2i \right)+1$};

        \draw (3.5, 16.5) -- (7, 16.5) node [right, black, xshift=1cm] 
        {\huge row $2 \left( \sum Q + K - 3 - i \right)+1$};

        \draw (3.5, 19.5) -- (7, 19.5) node [right, black, xshift=1cm] 
        {\huge row $2 \left( \sum Q + K - 1 - i \right)+1$};

        \draw (3.5, 24.5) -- (7, 24.5) node [right, black, xshift=1cm] 
        {\huge row $2 \left( \sum Q + K \right)$};
    \end{tikzpicture}
    }
    \caption{Depiction of invariants in phase 5}
    \label{fig:invariantsfive}
\end{figure}

Now consider when all blocks has been placed. Then we have $i = \sum M_w -2$ and the gameboard will appear like in~\autoref{fig:afterplace}. The interval in invariant 4 will completely disappear and the terrain will be completely checkered apart from the interval mentioned in invariant 3.

Now only three posibilites exist. We will consider how many cells can be cleared in each instance:

\begin{enumerate}
\item $K < \sum M_w$ 

    The bottom section is exhausted, therefore all the white cells in column 3 and all the black cells in column 4 are cleared. However, $\Delta = \sum M_w - K$ $\mathbf{LB}$ blocks will be placed which 
    \begin{itemize}
    \item Does not clear any black cells, since there aren't any left to match with in column 4.
    \item Does not clear any white cells, since there aren't left in column 3 to match those in column 4.
    Hence we have cleared $8 \left( \sum M_w-2-\Delta \right)$ cells.
    \end{itemize}
\item $K > \sum M_w$

    The bottom section is not exhausted. Therefore $\Delta = K - M_w$ $\mathbf{LB}$ will be placed which clears 4 black cells, but not 4 white cells. Hence we have cleared $8 \left( \sum M_w - 2 -\Delta \right) + 4 \Delta$ cells.
\item $K = \sum M_w$

    The bottom section is exhausted exactly when the final block is placed. This means that every placed $\mathbf{LB}$ block has cleared 4 black cells and 4 white cells. Hence we have cleared $8 \left( \sum M_w - 2 \right)$ cells.
\end{enumerate}

\begin{figure}[H]
    \centering
    \resizebox{!}{0.3\paperheight}{
    \begin{tikzpicture}
        \allplaced{0}{0}
        \draw[dashed] (-1, 0) -- (6, 0);
        \draw[dashed] (-1, 6) -- (6, 6);
        \draw[dashed] (-1, 11) -- (6, 11);
        \draw[dashed] (-1, 13) -- (6, 13);
        \draw[dashed] (-1, 19) -- (6, 19);
        \draw[dashed] (5, 0) -- (5, 19);
        \node at (0.5, -0.5) {\large 1};
        \node at (1.5, -0.5) {\large 2};
        \node at (2.5, -0.5) {\large 3};
        \node at (3.5, -0.5) {\large 4};
        \node at (4.5, -0.5) {\large 5};
        \draw [decorate,decoration={brace,amplitude=10pt},xshift=-12pt,yshift=0pt]
        (0, 0) -- (0, 6) node [left,align=right,black,midway,xshift=-1cm]
        {\huge $2 \left( K - M_w \right)$ rows};

        \draw [decorate,decoration={brace,amplitude=10pt},xshift=-12pt,yshift=0pt]
        (0, 6) -- (0, 11) node [left,align=right,black,midway,xshift=-1cm]
        {\huge $2 \left( \sum Q - 1 \right) +1$ rows};

        \draw [decorate,decoration={brace,amplitude=10pt},xshift=-12pt,yshift=0pt]
        (0, 13) -- (0, 19) node [left,align=right,black,midway,xshift=-1cm]
        {\huge $2 \sum M_w - 1$ rows};

        \draw (3.5, 0.5) -- (7, 0.5) node [right, black, xshift=1cm] 
        {\huge row 1};

        \draw (3.5, 5.5) -- (7, 5.5) node [right, black, xshift=1cm, yshift=-0.5cm] 
        {\huge row $2 \left( K-M_w \right)$};

        \draw (3.5, 6.5) -- (7, 6.5) node [right, black, xshift=1cm, yshift=0.5cm] 
        {\huge row $2 \left( K-M_w \right) + 1$};

        \draw (3.5, 10.5) -- (7, 10.5) node [right, black, xshift=1cm, yshift=-0.5cm] 
        {\huge row $2 \left( \sum Q + K - M_w - 1 \right)+1$};

        \draw (3.5, 13.5) -- (7, 13.5) node [right, black, xshift=1cm] 
        {\huge row $2 \left( \sum Q + K - M_w + 1 \right)$};

        \draw (3.5, 18.5) -- (7, 18.5) node [right, black, xshift=1cm] 
        {\huge row $2 \left( \sum Q + K \right)$};
    \end{tikzpicture}
    }
    \caption{Gameboard after each block has been placed}
    \label{fig:afterplace}
\end{figure}

\subsection{Summing the phases}

When all the phases are done, the maximum amount of cleared cells is equal to one of three equations. Let $\Delta = |\sum M_w - K|$, and $C$ be the maximum amount of cleared cells. The following is obtained:

\begin{equation} \tag{$K < \sum M_w$}
\begin{split}
C & = 8a + 5 \sum Q + 4 \sum M_w + 4 - 8 \Delta  \\
& = 8a + 5 \sum Q + 4 \sum M_w + 4 - 8 \left( \sum M_w - K\right) \phantom{+ 4K} \\
& = 8a + 5 \sum Q + 4 + 4 \left(2K - \sum M_w \right)
\end{split}
\end{equation}

\begin{equation} \tag{$K > \sum M_w$}
\begin{split}
C & = 8a + 5 \sum Q + 4 \sum M_w + 4 - 4 \Delta \\
& = 8a + 5 \sum Q + 4 \sum M_w + 4 - 4 \left( K - \sum M_w \right) \phantom{+ 4K} \\
& = 8a + 5 \sum Q + 4 + 4 \left(2 \sum M_w - K \right)
\end{split}
\end{equation}

\begin{equation} \tag{$K = \sum M_w$}
\begin{split}
C & = 8a + 5 \sum Q + 4 \sum M_w + 4 \phantom{- 8 \Delta} \\
& = 8a + 5 \sum Q + 4 + 4K \phantom{8 \left(\sum M_w + K \right)}
\end{split}
\end{equation}

In the case of $K < \sum M_w$, it is clear that $\left(2K - \sum M_w \right) < K$. In case of $K > \sum M_w$, is is clear that $\left(2 \sum M_w - K \right) < K$. Hence $8a + 5 \sum Q + 4 + 4K$ is the maximum amount of cleared cells in this game. This amount is only cleared when $K = \sum M_w$.

Since $K = \sum M_w$ if and only if there exists a solution $S = \{s_1, \ldots, s_b \} \subseteq Q, \sum S = K$ to the \textit{Subset sum} instance $\langle Q, K \rangle$, we have proved the the following theorem: \\

\begin{thm}
\label{thm:nphard}
The (offline, no-rotation, acyclic) k-cleared-cells problem is NP-hard.
\end{thm}
