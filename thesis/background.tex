\section{Background}

\subsection{Reduction}

\subsection{NP and PSPACE}
NP is the complexity class of all decision problems where a yes-instance can be verified in polynomial time. If it is possible to reduce all problems in NP to a certain problem, that problem is said to be NP-hard. A problem is referred to as a NP-complete problem if it is shown that it is both in NP and NP-hard. From this follows that if it is possible to reduce a NP-complete problem to some problem in NP using polynomial time, that problem is NP-complete as well.

PSPACE is the complexity class of all decision problems that can be solved by a Turing machine in a polynomial amount of memory space. Similar to NP-complete, a problem is PSPACE-complete if it is in PSPACE and all other problems in PSPACE can be reduced to the problem.

\subsection{The Subset Sum problem}

\subsection{Previous studies}

Several reports concerning computational complexity in video games have been published in recent years. During our research we have read some of these reports to get a good understanding what research has been made. These reports have covered games such as classical Nintendo games, Lemmings, Minesweeper and Bejeweled. These problems have been shown to be either NP-hard or PSPACE-hard and in most cases completeness in the respective complexity class has been proven.

One of the more interesting reports for our problem domain has been a report on the computational complexity of Tetris. In this report we learn that Tetris is NP-complete,  both when optimizing and approximating. Tetris is similar to Lumines in that it also consists of a grid-based game board and the main game mechanic is organizing sequences of pieces.

When showing NP-completeness in Tetris, 3-Partition instances were mapped to instances of a defined game board with sequences of pieces bound to the integers from the 3-Partition instances. Because Lumines has the similarities to Tetris explained above, this approach is interesting when considering a reduction to Lumines as well.

The existing research on Lumines that we have been able to find consists of a paper studying effective strategies. In this paper we learn that there exists strategies that can never lose, no matter what sequence of pieces are dropped. This is different to Tetris where losing can be inevitable if the computer is allowed to change the sequence according to how the player organizes the pieces. Another relevant difference the paper brings up is that in Lumines it is possible to create terrain that can not be cleared, while in Tetris a gameboard can always be cleared if the player is given an appropriate sequence of pieces.
