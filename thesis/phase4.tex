\subsubsection{Phase 4}
If the blocks has been placed as described in the previous phases, any block can only be placed in column 4-5 of the open well without trigger a game over. A sequence of $\sum Q - 1$ $\mathbf{LB}$ blocks are generated. Placing these blocks in column 4-5 of the open well will collapse column 4. This makes room for another placement, just like $\mathbf{H}$ blocks did in \nameref{subsub:phaseone}. Since no other terrain is cleared as a result of this placement, we are forced to continue placing the $\mathbf{LB}$ blocks in the jame manner.

The first $\sum Q - M_w$ blocks will clear 4 cells each, and gradually slide down the white cells of column 4.

When placing block $\sum Q - \sum M_w + 1$ the well open $w$ thus has the properties depicted in~\autoref{fig:beforecol}.

\begin{figure}[H]
    \centering
    \resizebox{!}{0.3\paperheight}{
    \begin{tikzpicture}
        \beforecollapse{0}{0}
        \lumlblack{3}{22}
        \draw[dashed] (-1, 0) -- (6, 0);
        \draw[dashed] (-1, 6) -- (6, 6);
        \draw[dashed] (-1, 12) -- (6, 12);
        \draw[dashed] (-1, 20) -- (6, 20);
        \draw[dashed] (5, 0) -- (5, 22);
        \node at (0.5, -0.5) {\large 1};
        \node at (1.5, -0.5) {\large 2};
        \node at (2.5, -0.5) {\large 3};
        \node at (3.5, -0.5) {\large 4};
        \node at (4.5, -0.5) {\large 5};
        \draw [decorate,decoration={brace,amplitude=10pt},xshift=-12pt,yshift=0pt]
        (0, 0) -- (0, 6) node [left,align=right,black,midway,xshift=-1cm]
        {\huge $2 \left( K-1 \right)$ rows};
        \draw [decorate,decoration={brace,amplitude=10pt},xshift=-12pt,yshift=0pt]
        (0, 6) -- (0, 12) node [left,align=right,black,midway,xshift=-1cm]
        {\huge $2 \sum M_w$ rows};
        \draw [decorate,decoration={brace,amplitude=10pt},xshift=-12pt,yshift=0pt]
        (0, 12) -- (0, 20) node [left,align=right,black,midway,xshift=-1cm]
        {\huge $2 \left( \sum Q - \sum M_w \right) $ rows};
        \draw (3.5, 0.5) -- (7, 0.5) node [right, black, xshift=1cm] 
        {\huge row 1};
        \draw (3.5, 5.5) -- (7, 5.5) node [right, black, xshift=1cm, yshift=-0.5cm] 
        {\huge row $2 \left( K-1 \right)$};
        \draw (3.5, 6.5) -- (7, 6.5) node [right, black, xshift=1cm, yshift=0.5cm] 
        {\huge row $2 \left( K-1 \right) + 1$};
        \draw (3.5, 11.5) -- (7, 11.5) node [right, black, xshift=1cm, yshift=-0.5cm] 
        {\huge row $2 \left( K-1+ \sum M_w \right)$};
        \draw (3.5, 12.5) -- (7, 12.5) node [right, black, xshift=1cm, yshift=0.5cm] 
        {\huge row $2 \left( K-1+ \sum M_w \right) +1 $};
        \draw (3.5, 19.5) -- (7, 19.5) node [right, black, xshift=1cm] 
        {\huge row $2 \left( \sum Q + K - 1 \right)$};
    \end{tikzpicture}
    }
    \caption{Gameboard state for the open well $w$}
    \label{fig:beforecol}
\end{figure}

The placement of block $\sum Q - \sum M_w + 1$, and the subsequent collapse of column 4 will result in 3 white cells from column 4 matching 3 white cells from column 3, thus clearing a $3 \times 2$ white rectangle. This will in turn result in the collapse of column 3 and 4 by three rows as depicted in~\autoref{fig:collapse}. 

The placement will indirectly clear 10 cells in total. Thus in this phase $4 \left( \sum Q - M_w \right) + 10$ cells can be cleared at most.

\begin{figure}[H]
    \centering
    \resizebox{!}{0.3\paperheight}{
    \begin{tikzpicture}
        \collapsepre{0}{0}
        \draw[dashed] (-1, 0) -- (5, 0);
        \draw[dashed] (-1, 6) -- (5, 6);
        \draw[dashed] (-1, 12) -- (5, 12);
        \draw[dashed] (-1, 20) -- (5, 20);
        \draw[dashed] (5, 0) -- (5, 22);
        \node at (0.5, -0.5) {\large 1};
        \node at (1.5, -0.5) {\large 2};
        \node at (2.5, -0.5) {\large 3};
        \node at (3.5, -0.5) {\large 4};
        \node at (4.5, -0.5) {\large 5};
        \draw[->, line width=10pt] (6, 11) -- (9, 11);
        \collapsepost{10}{0}
        \draw[dashed] (9, 0) -- (15, 0);
        \draw[dashed] (9, 6) -- (15, 6);
        \draw[dashed] (9, 12) -- (15, 12);
        \draw[dashed] (9, 20) -- (15, 20);
        \draw[dashed] (15, 0) -- (15, 22);
        \node at (10.5, -0.5) {\large 1};
        \node at (11.5, -0.5) {\large 2};
        \node at (12.5, -0.5) {\large 3};
        \node at (13.5, -0.5) {\large 4};
        \node at (14.5, -0.5) {\large 5};

    \end{tikzpicture}
    }
    \caption{Collapsing of well $w$}
    \label{fig:collapse}
\end{figure}

At the end of this phase, at most $8a + 5 \sum Q - 4 \sum M_w + 20$ cells have been cleared in total.
