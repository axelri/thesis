\subsection{Formal definitions}
\label{sub:formal}

This formalization of Lumines roughly follows the methodology of \citeauthor{tetris} in their paper on the computational complexity of Tetris \cite{tetris}. Please note that the formalization corresponds to the simplified model of Lumines considered in this report, rather than the actual video game. With this in mind, in this section `Lumines' should be read as `Lumines with the constraints and modifications this report assumes'.

\begin{description}[style=unboxed, leftmargin=0cm,labelsep=1em]
    \item[The gameboard] The \emph{gameboard} is a grid of $m$ rows and $n$ columns, indexed from bottom-to-top and left-to-right. The $\langle i,j \rangle$th \emph{cell} is either \emph{unfilled}, \emph{black} or \emph{white}. When a cell is either black or white, we call the cell \emph{filled}. In a legal Lumines gameboard, no cell is unfilled if some cell above it is filled.

    \item[Game blocks] In Lumines the player can only manipulate $2 \times 2$ squares of cells called \textit{blocks}. The attributes of a block in play is defined as a \emph{block state P}.It is 3-tuple, consisting of: 
    \begin{enumerate}
        \item \emph{block colors}, a 4-tuple $\in \{B,W\}^4$ corresponding to the corner colors of the block, counted clockwise from the lower-left corner. The six blocks core blocks pictured in \hyperref[fig:pieces]{figure \ref*{fig:pieces}} correspond to the following piece colors:

        \begin{description}
            \item[MW] $(W,W,W,W)$
            \item[MB] $(B,B,B,B)$
            \item[LW] $(W,B,W,W)$
            \item[LB] $(B,W,B,B)$
            \item[H] $(W,W,B,B)$
            \item[X] $(W,B,W,B)$
        \end{description}

        \item a \emph{position} of the block's lower-left corner on the gameboard, chosen from $\{1, \ldots, m-1\} \times \{1, \ldots, n-1\}$.
        \item the value \emph{fixed} or \emph{unfixed}.
    \end{enumerate}

In the \textit{initial block state}, the block is in its base orientation unless noted, is always unfixed and its initial position $\langle m-1, \lfloor n/2 \rfloor \rangle$.

    \item[Rotating blocks] Rotation of blocks is defined to be a transformation of the block's colors, according to the permutation $\sigma: \{B,W\}^4 \mapsto \{B,W\}^4$ corresponding to some rotation. The four possible permutations are:

    \begin{enumerate}
        \item $id$: $\sigma_{id}(\mathbf{c}) := (c_1)(c_2)(c_3)(c_4)$
        \item $\frac{\pi}{2}$: $\sigma_{C}(\mathbf{c}) := (c_1\;c_2\;c_3\;c_4)$
        \item $\pi$: $\sigma_{H}(\mathbf{c}) := (c_1\;c_3)(c_2\;c_4)$
        \item $\frac{3 \pi}{2} = -\frac{\pi}{2}$: $\sigma_{CC}(\mathbf{c}) := (c_1\;c_4\;c_3\;c_2)$.
    \end{enumerate}

    A shorthand is defined for aliasing the blocks in some rotation. For any block color $b \in \{B, W\}^4$ we define $b_{r} = \sigma_{r}(b)$. For example the monochromatic black block rotated one quarter-turn counterclockwise can be referred to as $\mathbf{MB}_{CC}$. Please note that the purpose of this definition is to aid in the proof. In this system rotation is not an allowed block manipulation during gameplay.

    \item[Game operations] No operations are legal for a piece $P = (\mathbf{c}, \langle i,j \rangle, fixed)$. The following operations are legal for a piece $P = (\mathbf{c}, \langle i,j \rangle, unfixed)$, with current gameboard $B$:

    \begin{enumerate}
        \item A \emph{slide to the right}. if the cells $\langle i,j+1 \rangle$ and $\langle i+1, j+1 \rangle$ are unfilled and in the bounds of the gamboard the move is legal. The new block state is $(\mathbf{c}, \langle i, j+1 \rangle, unfixed)$.
    \item A \emph{slide to the left}. if the cells $\langle i,j-1 \rangle$ and $\langle i+1, j-1 \rangle$ are unfilled and in the bounds of the gameboard, the move is legal. The new block state is $(\mathbf{c}, \langle i, j-1 \rangle, unfixed)$
\item A \emph{drop} by one row, if the cells $\langle i-1, j \rangle$ and $\langle i-1, j+1 \rangle$ are unfilled and in the bounds of the gameboard, the move is legal. The new block state is $(\mathbf{c}, \langle i-1, j \rangle, unfixed)$.
        \item A \emph{fix}. If the cells $\langle i-1, j \rangle$ or $\langle i-1, j+1 \rangle$ are filled or $i=1$ the move is legal. The new block state is $(\mathbf{c}, \langle i, j \rangle, fixed)$.
    \end{enumerate}

\item[Playing the game] A \textit{trajectory} $\tau$ of a block $P$ is a sequence of (legal) game moves starting from the initial block state and ending with a fix operation. The application of a trajectory to a block $P$ in a gameboard $B$ renders a new gameboard $B'$ according to the following rules:

    \begin{enumerate}
            \item $B'$ is initially $B$ with the cells of block $P$ filled.
            \item If the block is fixed, any column of the block which is not directly above a filled cell will continue to drop until it reaches either a filled cell or the bottom row. $B'$ is the result of this fix operation.
            \item If the block is fixed and the block position is $\langle m-1, j\rangle$, any $j \in \{1, \ldots, n-1\}$, a \textit{game over} is triggered.
            \item If any filled cell $\langle i,j \rangle$ in $B'$ and its surrounding cells $\langle i+1,j+1 \rangle$, $\langle i, j+1 \rangle$ and $\langle i+1, j \rangle$ are the same color, these cells are marked and cleared instantly. Any filled cells in $B'$ not directly above other filled cells or the bottom row now drops until they are. This stage is repeated until a further step would not result in any changes on the gameboard. $B'$ is now the results of these steps.
    \end{enumerate}

    A \textit{game} $(B_0, P_1, \ldots, P_p)$ is defined as an initial gameboard and a sequence of blocks to be placed by the player. A \textit{trajectory sequence} $\phi$ is a sequence $(B_0, \tau_1, B_1, \ldots ,\tau_p, B_p)$ which for each $i$ the trajectory $\tau_i$ applied to the game block $P_i$ on the gameboard $B_{i-1}$ results in gameboard $B_i$. If $\phi$ contains any trajectory $\tau_q$ which triggers a game over, $\phi$ naturally terminates at $B_q$ instead of $B_p$.
    \end{description}

\subsubsection{The k-cleared-cells problem}
The main Lumines goal this paper is concerned with is the problem of maximizing the amount of cleared cells given a particular game. Naturally the domain of the problem is the simplified version of Lumines previously presented in~\ref{sub:formal}. The problem is formulated as a decision problem in this domain as follows:

\begin{quote}
    \textbf{The (offline, no-rotation) k-cleared-cells problem}: Given a Lumines game $G$ and a goal $C \in \mathbb{N}$, does there exist a trajectory sequence $\phi$ such that when $\phi$ is applied to $G$, at least $C$ cells are cleared in total?
\end{quote}

\subsubsection{Acyclic goals}
The notion of acyclic goals is presented in \cite{tetris}. This definition will aid in proving k-cleared-cells to be in NP. It is used in a similar manner in the original paper.

\begin{quote}
    We say that an objective function $\Phi$ is \textit{acyclic} when, for all games $G$, if there is a trajectory sequence $\phi$ so that $\Phi(G, \phi)$ holds, then there is a trajectory sequence $\phi'$ so that $\Phi(G, \phi')$ holds and there are no repeated piece states in $\phi'$.
\end{quote}
