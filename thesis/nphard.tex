\subsection{Is k-cleared-cells NP-hard?}

We prove that k-cleared-cells is NP-hard by constructing a reduction from the \textit{Subset sum} problem. The reduction consists of two parts: one part which considers $K > 1$ in the given \textit{Subset sum} instance, and one part which considers the special case $K = 1$.

The goal of each part is to prove the following: 

\begin{quote}
Given a \textit{Subset sum} instance $P = \{p_1, \ldots, p_n\}, K \in \{1\} ,\{2, 3, \ldots \}$, there exists a solution $M = \{q_1, \ldots, q_m \} \subseteq P, \sum M = K$ if and only if there exists a trajectory sequence $\phi$ which clears $k$ cells in game $G$ in some instance of \textit{k-cleared-cells}.
\end{quote}

\subsubsection{The initial gameboard}

The initial gameboard $B_0$ in the constructed \textit{k-cleared-cells} instance is pictured in~\autoref{fig:initial}. It consists of two identical structures which will from here on be referred to as ``wells''. From the figure we can deduce that $B$ is a $2 \left( \sum P + K \right) \times 10$-sized gameboard.

\begin{figure}[H]
    \centering
    \resizebox{0.5\textwidth}{!}{
    \begin{tikzpicture}
        \welldefault{0}{0}
        \welldefault{5}{0}
        \draw[dashed] (0, -1) -- (0, 11);
        \draw[dashed] (9, -1) -- (9, 11);
        \draw[dashed] (-1, 0) -- (10, 0);
        \draw[dashed] (-1, 10) -- (10, 10);
        \draw [decorate,decoration={brace,amplitude=10pt},xshift=-12pt,yshift=0pt]
        (0, 0) -- (0, 10) node [align=right, black,midway,xshift=-4cm]
        {\Huge $2 \left( \sum P + K \right)$ rows};
        \draw [decorate,decoration={brace,amplitude=10pt,mirror},xshift=0pt,yshift=-12pt]
        (0, 0) -- (9, 0) node [black,midway,yshift=-1cm]
        {\Huge $10$ columns};
    \end{tikzpicture}
    }
    \caption{The initial gameboard}
    \label{fig:initial}
\end{figure}

\begin{figure}[H]
    \centering
    \resizebox{!}{0.3\paperheight}{
    \begin{tikzpicture}
        \welldetailed{0}{0}
        \draw[dashed] (-1, 0) -- (6, 0);
        \draw[dashed] (-1, 14) -- (6, 14);
        \draw[dashed] (-1, 6) -- (6, 6);
        \draw[dashed] (5, 0) -- (5, 16);
        \node at (0.5, -0.5) {\large 1};
        \node at (1.5, -0.5) {\large 2};
        \node at (2.5, -0.5) {\large 3};
        \node at (3.5, -0.5) {\large 4};
        \node at (4.5, -0.5) {\large 5};
        \draw [decorate,decoration={brace,amplitude=10pt},xshift=-12pt,yshift=0pt]
        (0, 0) -- (0, 6) node [align=right,black,midway,xshift=-3cm]
        {\huge $2 \left( K-1 \right)$ rows};
        \draw [decorate,decoration={brace,amplitude=10pt},xshift=-12pt,yshift=0pt]
        (0, 6) -- (0, 14) node [align=right,black,midway,xshift=-3cm]
        {\huge $2 \sum P$ rows};
        \draw (3.5, 0.5) -- (7, 0.5) node [black, xshift=2cm] 
        {\huge row 1};
        \draw (3.5, 6.5) -- (7, 6.5) node [black, xshift=3cm, yshift=-1.5cm]
        {\huge row $2 \left( K+1 \right)$};
        \draw (3.5, 5.5) -- (7, 5.5) node [black, xshift=3.5cm, yshift=1.5cm]
        {\huge row $2 \left( K+1 \right) +1$};
        \draw (3.5, 13.5) -- (7, 13.5) node [black, xshift=4cm]
        {\huge row $2 \left( K-1+ \sum P \right)$};
    \end{tikzpicture}
    }
    \caption{The well structure}
    \label{fig:wells}
\end{figure}
