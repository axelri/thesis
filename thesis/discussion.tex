\section{Discussion}
\label{discussion}

\subsection{Rotations}

The two main ways the player can control the Lumines game found on the PSP is to rotate and move the blocks left and right. A simplified model where the player is not allowed to rotate the blocks has been used in the reduction given in this report, meaning an important part of the game is not considered in the proof. It was planned to find a reduction which included rotations but because of time constraints this was not executed and a model without rotations was used instead. Intuitively rotation should have some impact on the hardness of the game. Because of the fact we found it harder to find a proper reduction of the Subset sum when a rotation model was used the possibility that Lumines is actually easier than NP-hard exist. However this does not have to be the case, rotations may as well be harder if it could be proven that an algorithm have to consider all possible ways to build terrain. During this project nothing has been found that tells us how rotations change the complexity and should therefore be examined to get a better understanding of the real game's complexity.

\subsection{Sweep-line}

In Lumines the sweep-line moves in constant speed from left to right. The marked squares are then cleared as the sweep-line scans the cells in their respective marked squares. The simplified Lumines model that is presented in this paper uses an instant sweep-line, in which the cells are marked and cleared simultaneously when a square is formed. This is similar to how rows are cleared in Tetris, but not the most accurate representation of how the sweep-line works in Lumines. There is probably room for improvement in making a more accurate model of the sweep-line game mechanic. For example, every $n\text{th}$ block that is fixed could trigger the clearing of marked cells. The less accurate model in this report was chosen since it was more suitable to work with given the time limit of the project. It is however important to note that the amount of cells that can be cleared in a given sequence can vary dramatically between different models of the sweep-line (see \textbf{todo figure showing this}). Therefore, in future work, inquires into the impact of the chosen sweep-line model on the computational complexity would be a promising area of research.

\subsection{Online Lumines}

In this report the offline version of Lumines has been considered. This is similar to previous studies on Match-three games and Tetris, where offline versions of the respective games have been considered. In these reports the authors argue that if the computational complexity of a property can be shown in the offline version of the games, intuitively it corresponding computational complexity in the online version should be as hard or harder. According to Demaine, Hoffman and Holzer, ``...it is only easier to play Tetris with complete knowledge of the future, so the difficulty of playing the offline version suggests the difficulty of playing the online version'' \cite[p. 2]{tetris}. If this statement is true it is probable that this is also the case for Lumines. However none of the examined papers present any proof that this is the case, either in general or specifically to their respective considered games. Therefore the online version of Lumines should be examined in the future to research if such a proof can be conducted.

\subsection{Subset sum}

None of the reductions encountered in examined material utilized the subset sum problem. The most commonly found problem to reduce from was the 3-SAT problem. It is the author's beliefs that this is the first paper concerning video game complexity using the subset sum problem. Although the nature of NP-completeness permits all NP-complete problem to be equally valid for use in reductions, some problems are inherently easier to reduce from than others depending on the problem to reduce to. The results from this report shows promising signs that subset sum is suitable problem to reduce from when examining other 2D stacking puzzle games. 
