\subsubsection{The special case $K = 1$}
The only possible solution to the \textit{Subset sum} instance $\langle Q = \{q_1, \ldots, q_a\}, 1 \rangle$ is $S = \{1\}$, since any other element of $Q$ will be too large. Knowing this, handling this special case is an easy task. A $2a+2 \times 2$ empty gameboard is created. Then any element $q_i$ is transformed into a $\mathbf{X}$ block if $q_i \not = 1$, or into an $\mathbf{MW}$ block if $q_i = 1$. Dropping the blocks and fixing them at the bottom is evidently the only allowed operation in this gameboard. Since the $\mathbf{X}$ blocks will always create alternating terrain, cells are only cleared when an $\mathbf{MW}$ block is placed. If an $\mathbf{MW}$ block is placed, 4 cells are cleared in total, otherwise no cells are cleared during the game.

Thus there is a trajectory sequence $\phi$ that clears at least 4 cells in $G$, if and only if the \textit{subset sum} instance $\langle Q, 1 \rangle$ has a solution. The reduction therefore holds for $K = 1$.

\begin{figure}[H]
    \centering
    \resizebox{!}{0.18\paperheight}{
    \begin{tikzpicture}
        \specialboard{0}{0}
        \draw[dashed] (-1, 0) -- (2, 0);
        \draw[dashed] (-1, 6) -- (2, 6);
        \draw[dashed] (-1, 8) -- (2, 8);
        \draw [decorate,decoration={brace,amplitude=10pt},xshift=-12pt,yshift=0pt]
        (0, 0) -- (0, 6) node [left,align=right,black,midway,xshift=-1cm]
        {\huge $2a$ rows};
        \draw [decorate,decoration={brace,amplitude=10pt},xshift=-12pt,yshift=0pt]
        (0, 6) -- (0, 8) node [left,align=right,black,midway,xshift=-1cm]
        {\huge $2$ rows};
        \draw [decorate,decoration={mirror,brace,amplitude=10pt},xshift=0pt,yshift=0pt]
        (0, 0) -- (2, 0) node [below,align=right,black,midway,yshift=-0.5cm]
        {\huge $2$ columns};

    \end{tikzpicture}
    }
    \caption{Initial gameboard for $K=1$}
\end{figure}

\begin{figure}[H]
    \centering
    \begin{subfigure}[b]{0.10\textwidth}
        \resizebox{\linewidth}{!}{
            \begin{tikzpicture}
            \lumx{0}{0}
            \end{tikzpicture}
        }
        \caption{$q_i \not = 1$}
    \end{subfigure}
    \hspace{0.02\textwidth}
    \begin{subfigure}[b]{0.10\textwidth}
        \resizebox{\linewidth}{!}{
            \begin{tikzpicture}
            \lummonowhite{0}{0}
            \end{tikzpicture}
        }
        \caption{$q_i = 1$}
    \end{subfigure}
    \caption{Block transformations for $q_i$}
\end{figure}
