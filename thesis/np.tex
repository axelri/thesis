\subsection{Is k-cleared-cells in NP?}

We begin by proving the following useful lemma:\\

\begin{lem}
\label{lem:legality}
The legality of a trajectory sequence $\phi=(B_0, \tau_1, B_1, \ldots ,\tau_p, B_p)$ in any game $G=(B_0, P_1, \ldots, P_p)$ is a checkable objective function.
\end{lem}

\begin{proof}
For every trajectory $\tau$ in $\phi$, deciding whether any operation in $\tau$ is legal is a check in constant time according to the rules in~\ref{subsub:operations}.

After a fix operation has been made, the gameboard enters a loop and changes according to the clear semantics defined in~\ref{subsub:gameplay}. This loop can run at most $|B| / 4$ times, since at least 4 cells has to be cleared each iteration, and no change can occur if there aren't any cells left to clear. Each iteration requires checking at most $4|B|$ cells for monochromatic $2 \times 2$-squares to mark. Thus the time to compute the new gameboard for each $\tau$ in $\phi$ is $\in O(|B|^2)$.

Checking that applying $\tau_i$ to $P_i$ in $B_{i-1}$ yields $B_i$ requires $|B|$ comparisons. 

In total $p$ new gameboards has to be generated and compared. Let the maximum amount of operations in any trajectory $\tau$ in $\phi$ be bounded by $C \in \mathbb{N}$. Thus we have that checking the validity of $\phi$ in $G$ can be done in $O(p \cdot C \cdot |B|^3) = \text{poly}(|G|, |\phi|)$. 
\end{proof}

We proceed by proving this modified theorem from \citeauthor{tetris} \cite{tetris}: \\

\begin{thm}
\label{thm:npobj}
For any checkable acyclic objective $\Phi$ in Lumines, $\Phi \in \text{NP}$.
\end{thm}

\begin{proof}
We are given a Lumines game $G = (B_0, P_1, \ldots, P_p)$ and a acyclic trajectory sequence $\phi = (B_0, \tau_1, B_1, \ldots ,\tau_p, B_p)$. Let $B_i$ be $m \times n$ gameboard.

Since $\phi$ is acyclic, each of its $p$ trajectories can not contain more than $(m-1)(n-1) + 1 \in O(|B|)$ states, unfixed once in each possible position and fixed in one final position. Thus $|\phi| \in O(p \cdot |B|) \subset O(|G|)$. This implies $\text{poly}(|G|, |\phi|) \in \text{poly}(|G|)$.

Checking the validity of the $\phi$ can be done in $\text{poly}(|G|, |\phi|) \in \text{poly}(|G|)$ according to lemma~\ref{lem:legality}.

Since $\Phi$ is checkable, we can then in time $\text{poly}(|G|, |\phi|) \in \text{poly}(|G|)$ verify if $\Phi(G, \phi)$ holds. Thus $\Phi \in \text{NP}$.
\end{proof}

Now we just have to prove the following lemma, and the original question will be answered as a direct result:\\

\begin{lem}
The objective k-cleared-cells is checkable and acyclic.
\end{lem}

\begin{proof}
The objective is acyclic because it only depends on fixed block state at the end of each trajectory, since the fix operation is the only way to trigger clearing of cells and thereby changes to the gameboard. The path taken in the trajectory is therefore irrelevant, as long as its legal. 

One can count the total amount of cleared cells simply by scanning the gameboard after each fix loop iteration, and then summing the results of each iteration. As presented in the proof for lemma~\ref{lem:legality}, this loop runs $O(|B|^2)$ times, and the counting of cleared cells can be done in $O(|B|)$. In total the summing can therefore be done in $O(|B|^3 \cdot |\phi|) = \text{poly}(|G|, |\phi|)$ time. Thus the objective is checkable.
\end{proof}

This naturally leads us to the following corollary:\\

\begin{cor}
The k-cleared-cells problem is in NP.
\end{cor}
